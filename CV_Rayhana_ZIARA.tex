\documentclass[10pt,a4paper,sans]{moderncv}
\usepackage[utf8]{inputenc}
\usepackage[francais]{babel}

% style options are 'casual' (default), 'classic', 'oldstyle' and 'banking'
\moderncvstyle{casual}
\moderncvcolor{orange}
\usepackage[top=1.5cm, bottom=2cm, left=2cm, right=2cm]{geometry}
% Largeur de la colonne pour les dates
\setlength{\hintscolumnwidth}{2.5cm}

\name{Rayhana}{ZIARA}
%\title{Ingénieure R\&D en Calcul Scientifique}
\address{toto}{toto}
\phone[mobile]{00~00~00~00~00}
\email{rayhana.ziara@gmail.com}
\homepage{www.rayhana-ziara.com}
\social[github]{rayahz}
\social[linkedin][www.linkedin.com/in/rayhana-ziara-81713584]{Rayhana ZIARA}
%\quote{Stagiaire chez ANDRA}

\begin{document}
\makecvtitle

\section{Formations}
\cventry{2014--2016}{Master}{Université de Versailles Saint-Quentin-en-Yvelines (UVSQ)}{}{\textit{Diplômée}}{Spécialité Informatique Haute Performance et Simulation}
\cventry{2013--2014}{Licence Professionnelle}{IUT Orsay - Université Paris Sud XI}{}{\textit{Diplômée, Major de promo}}{Spécialité Programmation et Environnement Réparti}
\cventry{2011--2013}{DUT Informatique}{IUT Orsay - Université Paris Sud XI}{}{\textit{Diplômée}}{}
\cventry{2009--2011}{PAES}{Université Paris Sud XI}{}{\textit{Non diplômée}}{}
\cventry{2008--2009}{PCS0}{Université Paris Sud XI}{}{\textit{Diplômée}}{}
\cventry{Juin 2008}{Baccalauréat}{Lycée Léonard de Vinci}{Saint Michel sur Orge (91)}{\textit{Mention Bien}}{Série technologique, spécialité Sciences Medico Sociales}

\section{Expériences professionnelles}
\cventry{Mars 2016\\à Sept. 2016}
{STAGE - Ingénieur Calcul Scientifique}
{\href{http://andra.fr/}{ANDRA}}{}{}
{Simulation d'écoulements diphasiques et de transfert en milieu poreux
%Le stage a porté sur la simulation d'écoulements diphasiques et du transfert en milieux poreux. L'objectif a été de consolider et de finaliser la parallélisation du logiciel pour le modèle d'écoulements et de transport hydraulique insaturé à l'aide des librairies Hypre et Maphys.
\begin{itemize}
\item Analyse de la version saturée du code parallèle
\item Implémentation en Fortran 90 et tests de la version insaturée
\item Validation et tests de performance
\item Tests de scalabilité au CINES
\end{itemize}}

\cventry{Mai 2015\\à Sept. 2015}
{STAGE - Ingénieur Recherche}
{\href{http://www.exascale-computing.eu/}{Exascale Computing Research}}{}{}
{Extraire et rejouer un hotspot à partir d'une application MPI (framework open-source \href{https://benchmark-subsetting.github.io/cere/}{CERE})
%Le stage a porté sur le framework open-source CERE. Ce framework peut extraire et rejouer un hotspot à partir d'une application. Mon objectif a été de trouver un moyen d'extraire des hotspots à partir d'applications parallèle MPI. Pour cela j'ai dû :
\begin{itemize}
\item Isoler un processus
\item Wrapper les fonctions MPI
\item Capturer les communications MPI
\item Développement en Python et C
\item Tests effectués sur benchmarks NPB MPI 3.3
\end{itemize}}

\cventry{Sept. 2013\\à Sept. 2014}{APPRENTISSAGE - Analyste Développeur}{\href{http://securities.bnpparibas.com/fr/}{BNP Paribas Securities Services}}{}{}{Mise en place d'un support dédié aux reportings
%Dans le cadre de l'amélioration du support aux reportings, des écrans de paramétrage et de lancement ont été créés afin d'éviter certaines tâches manuellement effectuées par les équipes techniques.
%\begin{itemize}
%\item Étude de l'existant
%\item Écriture des spécifications fonctionnelles et techniques
%\item Développement en PL / SQL
%\item Tests et livraisons en plusieurs lots
%\end{itemize}
}

\cventry{Avril 2013\\à Juillet 2013}{STAGE - Développeur J2EE}{\href{https://www.societegenerale.fr/}{Société Générale}}{}{}{Mise en place d'outils IHM en formats standards PDF 
%\begin{itemize}
%\item Étude de l'existant
%\item Dossiers de conception
%\item Développement en J2EE et COBOL
%\end{itemize}
}

\section{Compétences en informatique}
\cvitem{Web}{HTML5/CSS3, PHP5, Javascript, Bootstrap}
\cvitem{Langages}{C, C++, Python, Fortran 90, Bash, \LaTeX, PL/SQL}
\cvitem{Librairies}{BLAS, LAPACK, VTK, OpenMP, MPI, Cuda}
\cvitem{SGBD}{MySQL, Oracle, PhpMyAdmin}
\cvitem{Systèmes}{Windows XP/Seven/8, Linux Ubuntu, Debian}
\cvitem{Logiciels}{Tortoise SVN, Git, Gnuplot, Paraview, Vim}
\cvlanguage{Anglais}{Lu, écrit, parlé -- TOEIC 890 / 990}{}

\section{Projets universitaires}
\cvlistitem{Méthode des itérations simultanées -- C}
\cvlistitem{Implémentation d'une factorisation ILU avec préconditionneur -- Octave, C, MPI}
\cvlistitem{Site web de support informatique -- HTML, PHP}
\cvlistitem{Jeux de tri -- HTML5, CSS3, Javascript}
\cvlistitem{Site e-commerce -- J2EE}

\section{Centres d'intérêts}
\cvitem{}{Lectures, photographies, voyages, Rubik's cube}
\cvitem{}{Nouvelles technologies, développement web}

\newpage

\name{Rayhana}{ZIARA}
%\title{R\&D Engineer in Scientific Data}
%\quote{Looking for an internship in HPC for a 6 months period}

\makecvtitle

\section{Education}
\cventry{2014--2016}{Master of Science}{Université de Versailles Saint-Quentin-en-Yvelines (UVSQ)}{}{\textit{Graduate}}{Speciality High Performance Computing \& Simulation}
\cventry{2013--2014}{Licence Professionnelle}{IUT Orsay - Université Paris Sud XI}{}{\textit{Graduate, Major}}{Professional Degree in Programming and Distributed Environment}
\cventry{2011--2013}{University Degree in Science Computer}{IUT Orsay - Université Paris Sud XI}{}{\textit{Graduate}}{}
\cventry{2009--2011}{First year of medical school (PAES)}{Université Paris Sud XI}{}{\textit{Not graduate}}{}
\cventry{2008--2009}{PCS0}{Université Paris Sud XI}{}{\textit{Graduate}}{Preparatory year for students that did not graduate high school in sciences}
\cventry{June 2008}{Medical Sciences Baccalaureate}{Lycée Léonard de Vinci}{St Michel sur Orge (91)}{\textit{With honors}}{}

\section{Professional activities}
\cventry{March 2016\\à Sept. 2016}
{INTERNSHIP - Scientific Data Engineer}
{\href{http://andra.fr/}{ANDRA}}{}{}
{Simulation of a flow and a transfer in porous media
\begin{itemize}
\item Study of the saturate version
\item Development and tests of the insaturate version
\item Validation and performance testing
\item Scalabilty tests at the CINES
\end{itemize}}

\cventry{May 2015\\to Sept. 2015}{INTERNSHIP - Research Engineer}{\href{http://www.exascale-computing.eu/}{Exascale Computing Research}}{}{}{Extract and replay hotspot from a MPI application (framework open-source \href{https://benchmark-subsetting.github.io/cere/}{CERE})
\begin{itemize}
\item Isolate one process
\item Wrapper MPI functions
\item Catch MPI communications
\item Development in Python and C
\item Tests made on the benchmarks NPB MPI 3.3
\end{itemize}}
\cventry{Sept. 2013\\to Sept. 2014}{APPRENTICESHIP - Analyst / Developer}{\href{http://securities.bnpparibas.com/fr/}{BNP Paribas Securities Services}}{}{}{Improvement of reporting support by creating setting and launching screens.
%\begin{itemize}
%\item Analysis and study of the existing
%\item Design conception (including technical architecture)
%\item Development (PL/SQL, Java and Oracle report)
%\item Unit tests and Production delivery
%\end{itemize}
}
\cventry{April 2013\\to July 2013}{INTERNSHIP - J2EE Developer}{\href{https://www.societegenerale.fr/}{Société Générale}}{}{}{Development of a new tool for viewing and printing digital images of checks. 
%\begin{itemize}
%\item Analysis and study of the existing
%\item Design conception (including technical architecture)
%\item Development (COBOL and J2EE)
%\end{itemize}
}

\section{Computing skills}
\cvitem{Web}{HTML5/CSS3, PHP5, Javascript, Bootstrap}
\cvitem{Languages}{C, C++, Python, Fortran 90, Bash, \LaTeX, PL/SQL}
\cvitem{Libraries}{BLAS, LAPACK, VTK, OpenMP, MPI, Cuda}
\cvitem{Database}{MySQL, Oracle, PhpMyAdmin}
\cvitem{OS}{Windows XP/Seven/8, Linux Ubuntu, Debian}
\cvitem{Hardware}{Tortoise SVN, Git, Gnuplot, Paraview, Vim}
\cvlanguage{English}{Read, written and spoken -- TOEIC 890 / 990}{}

\section{Projects made during my education}
\cvlistitem{Simultaneous iteration method -- C}
\cvlistitem{Implementation of an ILU factorization with a preconditioner -- Octave, C, MPI}
\cvlistitem{Support customer web service -- HTML, PHP}
\cvlistitem{Sorting game figures in ascending and descending order -- HTML5, CSS3, Javascript}
\cvlistitem{E-shop -- J2EE}

\section{Hobbies}
\cvitem{}{Reading, photography, traveling, Rubik's cube}
\cvitem{}{New technologies, web development}
\end{document}
